%%%%%%%%%%%%%%%%%%%%%%%%%%%%%%%%%%%%%%%%%%%%%%%%%%%%%%%%%%%%%%%%%%%%%%%%%%%%%%%%
% Medium Length Graduate Curriculum Vitae
% LaTeX Template
% Version 1.2 (3/28/15)
%
% This template has been downloaded from:
% http://www.LaTeXTemplates.com
%
% Original author:
% Rensselaer Polytechnic Institute 
% (http://www.rpi.edu/dept/arc/training/latex/resumes/)
%
% Modified by:
% Daniel L Marks <xleafr@gmail.com> 3/28/2015
%
% Important note:
% This template requires the res.cls file to be in the same directory as the
% .tex file. The res.cls file provides the resume style used for structuring the
% document.
%
%%%%%%%%%%%%%%%%%%%%%%%%%%%%%%%%%%%%%%%%%%%%%%%%%%%%%%%%%%%%%%%%%%%%%%%%%%%%%%%%

%-------------------------------------------------------------------------------
%	PACKAGES AND OTHER DOCUMENT CONFIGURATIONS
%-------------------------------------------------------------------------------

%%%%%%%%%%%%%%%%%%%%%%%%%%%%%%%%%%%%%%%%%%%%%%%%%%%%%%%%%%%%%%%%%%%%%%%%%%%%%%%%
% You can have multiple style options the legal options ones are:
%
%   centered:	the name and address are centered at the top of the page 
%				(default)
%
%   line:		the name is the left with a horizontal line then the address to
%				the right
%
%   overlapped:	the section titles overlap the body text (default)
%
%   margin:		the section titles are to the left of the body text
%		
%   11pt:		use 11 point fonts instead of 10 point fonts
%
%   12pt:		use 12 point fonts instead of 10 point fonts
%
%%%%%%%%%%%%%%%%%%%%%%%%%%%%%%%%%%%%%%%%%%%%%%%%%%%%%%%%%%%%%%%%%%%%%%%%%%%%%%%%

\documentclass[margin]{res}  

% Default font is the helvetica postscript font
\usepackage{helvet}
\usepackage{graphicx}
\usepackage{hyperref}
\hypersetup{hidelinks}

\usepackage[utf8x]{inputenc}

% Increase text height
\textheight=700pt

\newenvironment{absolutelynopagebreak}
  {\par\nobreak\vfil\penalty0\vfilneg
   \vtop\bgroup}
  {\par\xdef\tpd{\the\prevdepth}\egroup
   \prevdepth=\tpd}



\begin{document}

%-------------------------------------------------------------------------------
%	NAME AND ADDRESS SECTION
%-------------------------------------------------------------------------------
\name{Konstantinos Kallas}

% Note that addresses can be used for other contact information:
% -phone numbers
% -email addresses
% -linked-in profile

% \address{Mpoumpoulinas 27 \\Dionisos, Greece, 14576\\\\Address 1 line 3}
% \address{Address 2 line 1\\Address 2 line 2\\Address 2 line 3}

% Uncomment to add a third address
%\address{Address 3 line 1\\Address 3 line 2\\Address 3 line 3}
%-------------------------------------------------------------------------------

\begin{resume}

%-------------------------------------------------------------------------------
%	EDUCATION SECTION
%-------------------------------------------------------------------------------
\section{Contact} 
%% Address: Mpoumpoulinas 27, Dionisos, Greece, 14576 \\
%% Phone: +30 6957814666 \\
Email: \href{mailto:kallas@seas.upenn.edu}{kallas@seas.upenn.edu} -- Website: \href{https://angelhof.github.io/}{angelhof.github.io}
%% Github: \href{https://github.com/angelhof}{angelhof} \\

\section{Research}
The goal of my research is to enable ordinary developers to develop high-performance applications with robust correctness guarantees. To achieve this goal, I develop novel programmable software systems and enhance existing ones by reimagining their capabilities. I use formal methods and models to prove properties of these systems and their expected behavior.

\section{Education}
\textbf{University of Pennsylvania} \hfill {\em September 2018 -- present}\\
Computer and Information Science, PhD student \\
Advisor: Prof. Rajeev Alur

% \textbf{Eighth Summer School on Formal Techniques} \hfill \mbox{\em May 2018}\\
% Menlo College, Atherton, CA \\
% Organised by SRI International

\textbf{National Technical University of Athens} \hfill \mbox{\em October 2012 -- February 2018}\\
Electrical and Computer Engineering, Diploma \\ %% \hfill {\em GPA: 8.37/10} \\
% Specialization: Computer Science \hfill {\em Major (Last 2 years) GPA: 9.05/10} \\
Thesis: \textit{``HiPErJiT: A Profile-Driven Just-in-Time Compiler for Erlang''} \\
Advisor: Prof. Kostis Sagonas 

% Removing for now. Doesn't feel relevant for job apps.
% \textbf{Universita degli Studi dell'Aquila} \hfill {\em February 2015 -- July 2015} \\
% Computer Science, Erasmus Mobility Program %% \hfill {\em GPA: 9.4/10} \\
% Selected Courses: Programming Languages, Cryptography, and Software Engineering

% \textbf{Dionisos 1st High School, Greece} \hfill {\em Graduation: July 2012} \\
% Greek Apolitirion, ranked first among 120 students \hfill {\em GPA: 19.4/20}



\section{Employment}

\textbf{Research Intern}  \hfill {\em Summer 2020} \\
Microsoft Research, Redmond, US \\
Internship in the RiSE group; advised by Sebastian Burckhardt. \\
Worked on Durable Functions, a programming model for serverless applications.

\textbf{Research Intern}  \hfill {\em Summer 2019} \\
Amazon Web Services, New York, US \\
Internship in the Automated Reasoning Group; advised by Daniel Schwartz-Narbonne. \\
Worked on the verification of critical C code.

% \textbf{Google Summer of Code Student}  \hfill {\em May 2017 - September 2017} \\
% Worked with BEAM Community to extend the \textit{ejabberd} open source project. \\
% Implemented support for the \textit{``Let's Encrypt'' ACME} certificate acquiring protocol.

\textbf{Big Data Application Developer}  \hfill {\em Summer 2016} \\
Everis, Barcelona, Spain \\
Internship at the Big Data Center of Excellence. \\
Developed Big Data Applications using tools in the Hadoop ecosystem.

\section{Honors and Awards}

\textbf{A.G. Leventis Foundation PhD Grant} \hfill {\em 2021-2023}

\textbf{ACM SRC Grand Finals} \hfill {\em 2021} \\
\textit{2nd place} among SRC winners across all ACM conferences.

\textbf{HotOS 2021 Distinguished Presentation Award} \hfill {\em 2021} \\
Awarded for ``Unix Shell Programming: The Next 50 Years''.

\textbf{EuroSys 2021 Best Paper Award} \hfill {\em 2021} \\
Awarded for ``PaSh: Light-touch Data-Parallel Shell Processing''.

\textbf{POPL Student Research Competition} \hfill {\em 2021} \\
\textit{1st place} at the graduate category of the research competition. \\
Presented work on a parallelizing JiT compiler for shell scripts. 

\textbf{Gerondelis Foundation PhD Award} \hfill {\em 2020}

% \textbf{Programming Competitions}  \hfill {\em 2015-2018} \\
% Participation in many programming competitions. Notable examples: \\
% ICFP Programming Contest 2018 (Lightning) \textit{10th out of 91 teams}, \\
% IEEE Xtreme 2017 \textit{Top 5\%} and \textit{3rd} in Greece, IEEE Xtreme 2016 \textit{Top 10\%}

% \textbf{Heterogenous Computing Student Challenge Certificate}  \hfill {\em 2017} \\
% HiPEAC CSW, Zagreb, Croatia \\
% Optimizing GPU implementation of the K-means algorithm (NTUA-team) \\
% Supervision: Prof. Georgios Goumas

% \textbf{EESTech Challenge}  \hfill {\em 2017} \\ 
% Supervised Machine Learning Hackathon \\
% \textit{Joint 1st place} among 40 teams.

% \textbf{The Great Moment of Education Scholarship from Eurobank EFG}  \hfill {\em 2012} \\ 
% Achieving the highest rank in national qualifications exams in Dionisos high school.

% \textbf{9th European Union Science Olympiad} \hfill {\em 2011} \\ 
% Team-based science competition in Biology, Chemistry, and Physics.\\
% \textit{1st place} in local round and \textit{3rd place} in  national round. 

% \textbf{Mathematical Competition, Hellenic Mathematical Society} \hfill {\em 2010} \\ 
% Mathematical competition for high school students. \\
% Distinction in the 1st and 2nd local round. 


\section{Publications}

\begin{minipage}{\textwidth}
\textbf{Practically Correct, Just-in-Time Shell Script Parallelization}. \\
Konstantinos Kallas, Tammam Mustafa, Jan Bielak, Dimitris Karnikis, Thurston H.Y. Dang, Michael Greenberg, and Nikos Vasilakis. \\
16th USENIX Symposium on Operating Systems Design and Implementation (OSDI 22).
\end{minipage}

\begin{minipage}{\textwidth}
\textbf{Netherite: Efficient Execution of Serverless Workflows}. \\
Sebastian Burckhardt, Badrish Chandramouli, Chris Gillum, David Justo, Konstantinos Kallas, Connor McMahon, Christopher S. Meiklejohn, and Xiangfeng Zhu. \\
Proceedings of the VLDB Endowment (VLDB 2022).
\end{minipage}

\begin{minipage}{\textwidth}
\textbf{Stream Processing with Dependency-Guided Synchronization}. \\
Konstantinos Kallas$^*$, Filip Niksic$^*$, Caleb Stanford$^*$, and Rajeev Alur. \\
Proceedings of the 27th ACM SIGPLAN Symposium on Principles and Practice of Parallel Programming (PPoPP 2022).
\end{minipage}

\begin{minipage}{\textwidth}
\textbf{Charon: A Framework for Microservice Overload Control}. \\
Jiali Xing, Max Demoulin, Konstantinos Kallas, and Benjamin C. Lee. \\
Proceedings of the 18th ACM Workshop on Hot Topics in Networks (HotNets 2021).
\end{minipage}

\begin{minipage}{\textwidth}
\textbf{Durable Functions: Semantics for Stateful Serverless}. \\
Sebastian Burckhardt, Chris Gillum, David Justo, Konstantinos Kallas, Connor McMahon, and Christopher S. Meiklejohn. \\
Proceedings of the ACM on Programming Languages (OOPSLA 2021).
\end{minipage}

\begin{minipage}{\textwidth}
\textbf{An Order-aware Dataflow Model for Parallel Unix Pipelines}. \\
Shivam Handa$^*$, Konstantinos Kallas$^*$, Nikos Vasilakis$^*$, and Martin Rinard. \\
Proceedings of the ACM on Programming Languages (ICFP 2021).
\end{minipage}

\begin{minipage}{\textwidth}
\textbf{Synchronization Schemas}. \\
Rajeev Alur, Phillip Hillard, Zachary G. Ives, Konstantinos Kallas, Konstantinos Mamouras, Filip Niksic, Caleb Stanford, Val Tannen, and Anton Xue. \\
Invited Paper at Proceedings of the 40th Symposium on Principles of Database Systems (PODS 2021).
\end{minipage}

\begin{minipage}{\textwidth}
\textbf{Unix Shell Programming: The Next 50 Years}. \\
Michael Greenberg$^*$, Konstantinos Kallas$^*$, and Nikos Vasilakis$^*$. \\
Proceedings of the Workshop on Hot Topics in Operating Systems (HotOS 2021).\\
 \emph{Distinguished Presentation Award.}
\end{minipage}

\begin{minipage}{\textwidth}
\textbf{The Future of the Shell: Unix and Beyond}. \\
Michael Greenberg$^*$, Konstantinos Kallas$^*$, and Nikos Vasilakis$^*$. \\
Panel at the Workshop on Hot Topics in Operating Systems (HotOS 2021).
\end{minipage}

\begin{minipage}{\textwidth}
\textbf{PaSh: Light-touch Data-Parallel Shell Processing}. \\
Nikos Vasilakis$^*$, Konstantinos Kallas$^*$, Konstantinos Mamouras, Achilleas Benetopoulos, and Lazar M. Cvetković. \\
Proceedings of the Sixteenth European Conference on Computer Systems (EuroSys 2021).\\
 \emph{Best Paper Award.}
\end{minipage}

\begin{minipage}{\textwidth}
\textbf{Preventing Dynamic Library Compromise on Node. js via RWX-Based Privilege Reduction}. \\
Nikos Vasilakis, Cristian-Alexandru Staicu, Grigoris Ntousakis, Konstantinos Kallas, Ben Karel, André DeHon, and Michael Pradel. \\
Proceedings of the ACM SIGSAC Conference on Computer and Communications Security (CCS’21).
\end{minipage}

\begin{minipage}{\textwidth}
\textbf{Code-level model checking in the software development workflow at Amazon Web Services}. \\
Nathan Chong, Byron Cook, Jonathan Eidelman, Konstantinos Kallas, Kareem Khazem, Felipe R. Monteiro, Daniel Schwartz-Narbonne, Serdar Tasiran, Michael Tautschnig, and Mark R. Tuttle. \\
Software: Practice and Experience 2021.
\end{minipage}

\begin{minipage}{\textwidth}
\textbf{DiffStream: Differential Output Testing for Stream Processing Programs}. \\
Konstantinos Kallas$^*$, Filip Niksic$^*$, Caleb Stanford$^*$, and Rajeev Alur. \\
Proceedings of the ACM on Programming Languages (OOPSLA 2020).
\end{minipage}

\begin{minipage}{\textwidth}
\textbf{Code-Level Model Checking in the Software Development Workflow}. \\
Nathan Chong, Byron Cook, Konstantinos Kallas, Kareem Khazem, Felipe R. Monteiro, Daniel Schwartz-Narbonne, Serdar Tasiran, Michael Tautschnig, and Mark R. Tuttle. \\
42st International Conference on Software Engineering: Software Engineering in Practice (ICSE-SEIP 2020).
\end{minipage}

\begin{minipage}{\textwidth}
\textbf{Security Criteria for a Transparent Encryption Layer}. \\
Konstantinos Kallas, Clara Schneidewind, Benjamin C. Pierce, and Steve Zdancewic. \\
Workshop on Foundations of Computer Security (FCS 2019).
\end{minipage}

\begin{minipage}{\textwidth}
\textbf{HiPErJiT: A Profile-Driven Just-in-Time Compiler for Erlang}. \\
Konstantinos Kallas and Konstantinos Sagonas. \\
30th Symposium on Implementation and Application of Functional Languages (IFL 2018).
\end{minipage}




Notes: $^*$ indicates equal contribution. Citation counts exported from Google Scholar.

\section{Open Source Software}

\textbf{PaSh} (\href{https://github.com/binpash/pash}{Github: binpash/pash}) Stars: 510, Forks: 36 \\
A bolt-on system that automatically parallelizes arbitrary shell programs with theoretical and practical correctness guarantees. \\
{\em Hosted by the \href{https://www.linuxfoundation.org/press-release/linux-foundation-to-host-the-pash-project-accelerating-shell-scripting-with-automated-parallelization-for-industrial-use-cases/}{Linux Foundation}}.

\textbf{try} (\href{https://github.com/binpash/try}{Github: binpash/try}) Stars: 4885, Forks: 64 \\
A tool that lets you run a command and inspect its effects before committing them to your system.

\textbf{DiSh} (\href{https://github.com/binpash/dish}{Github: binpash/dish}) Stars: 15, Forks: 3 \\
A system that automatically scales out shell scripts that operate on files in HDFS.

\textbf{mu2sls} (\href{https://github.com/eniac/mu2sls}{Github: eniac/mu2sls}) Stars: 9, Forks: 4 \\
A framework for correctly implementing stateful microservice applications on serverless using standard Python. 

\textbf{Flumina} (\href{https://github.com/angelhof/flumina}{Github: angelhof/flumina}) Stars: 12, Forks: 1 \\
A programming model and system for stateful distributed streaming computations.

\textbf{DiffStream} (\href{https://github.com/fniksic/diffstream}{Github: fniksic/diffstream}) Stars: 6, Forks: 1 \\
A differential testing library for stream processing applications in Apache Flink.

\section{Press}

\textbf{Practically Correct, Just-in-Time Shell Script Parallelization} (\href{https://disseminatepodcast.podcastpage.io/episode/konstantinos-kallas-practically-correct-just-in-time-shell-script-parallelization-20}{link}) \\
Disseminate Podcast Episode 20, hosted by Jack Waudby. January 2023.

\textbf{Faster computing results without fear of errors} (\href{https://news.mit.edu/2022/faster-unix-computing-program-0607}{link}) \\
MIT News Article, written by Adam Zewe. June 2022.

\textbf{Linux Foundation to Host the PaSh Project, Accelerating Shell Scripting with Automated Parallelization for Industrial Use Cases} (\href{https://www.linuxfoundation.org/press/press-release/linux-foundation-to-host-the-pash-project-accelerating-shell-scripting-with-automated-parallelization-for-industrial-use-cases}{link}) \\
Linux Foundation Press Release, written by Kristin OConnell. September 2021.


\section{Research Mentoring}

\textbf{Dimitra Leventi} (NTUA, BSc) \hfill {\em 2023 -- present} \\
Characterization of shell workloads and development of a benchmark suite.

\textbf{Nikos Pagonas} (NTUA, BSc) \hfill {\em 2023 -- present} \\
Design and development of a serverless shell.

\textbf{Spyros Pavlatos} (UPenn, PhD) \hfill {\em 2022 -- present} \\
Development of correctness criteria for microservice applications.

\textbf{Akis Giannoukos} (UPenn, PhD) \hfill {\em 2022 -- present} \\
Overload control for microservice applications.

\textbf{Giorgos Liargovas} (AUEB, BSc) \hfill {\em 2022 -- present} \\
Out-of-order execution of shell scripts (paper at HotOS 2023).

\textbf{Tianyu (Eric) Zhu} (Stevens, BSc) \hfill {\em 2022 -- present} \\
Design and development of \texttt{try}, a lightweight isolation tool for Linux (over 4k stars on \href{https://github.com/binpash/try}{Github}).

\textbf{Jiali Xing} (UPenn, PhD) \hfill {\em 2021 -- present} \\
Overload control for microservice applications (paper at HotNets 2021).

\textbf{Tammam Mustafa} (MIT, BSc $\rightarrow$ Google) \hfill {\em 2021 -- 2023} \\
Design and development of DiSh (papers at OSDI 2022 and NSDI 2023).

\textbf{Achilles Benetopoulos} (NTUA, BSc $\rightarrow$ UCSC, PhD) \hfill {\em 2019 -- 2021} \\
Development of PaSh's runtime and benchmarking of shell programs (paper at EuroSys 2021). 

\textbf{Lazar Cvetkovic} (University of Belgrade, BSc $\rightarrow$ ETH, PhD) \hfill {\em 2019 -- 2021} \\
Specification framework for POSIX and GNU Coreutils commands (paper at EuroSys 2021). 


\section{Outreach}

\textbf{CS PhD MentoRes}  \hfill {\em 2021 -- present} \\
Co-organizer of mentoring initiative for students that are interested in applying for PhD programs in CS but lack adequate resources. We have provided mentoring and resources to more than 40 students since the initiative's start.

\textbf{SIGPLAN-M}  \hfill {\em 2021 -- present} \\
Participating mentor for students in the programming languages community.

\textbf{SOSP Mentoring}  \hfill {\em 2023} \\
Student mentor in SOSP 2023.

\section{Service}

\textbf{POPL 2023 Student Volunteer Co-Chair}  \hfill {\em 2023}

\textbf{OOPSLA 2023 External Review and Artifact Evaluation Committee}  \hfill {\em 2023}

\textbf{POPL 2022 Student Volunteer Co-Chair}  \hfill {\em 2022}

\textbf{HotOS 2021 Co-organizer of a panel on the future of the shell} (\href{https://fut-shell.github.io/}{link})  \hfill {\em 2021}

\textbf{VMCAI 2021 Artifact Evaluation Committee}  \hfill {\em 2021}

\textbf{POPL 2020 External Reviewer}  \hfill {\em 2020}

\section{Teaching Experience}

\textbf{Teaching Assistant}  \hfill {\em Fall 2021} \\
Institution: University of Pennsylvania \\
Course: \textit{Computer-Aided Verification}, Graduate level \\
Professor: Rajeev Alur

\textbf{Teaching Assistant}  \hfill {\em Fall 2019} \\
Institution: University of Pennsylvania \\
Course: \textit{Software Foundations}, Graduate level \\
Professor: Benjamin Pierce

\textbf{Lab Assistant}  \hfill {\em Fall 2017} \\
Institution: National Technical University of Athens \\
Course: \textit{Introduction to Programming}, Undergraduate level \\
Professors: S. Zachos, N. Papaspyrou, V. Kantere, and P. Potikas


\section{Invited Talks}

\begin{minipage}{\textwidth}
\textbf{PaSh: Practically Correct, Just-in-Time Shell Script Parallelization}. \hfill {\em 2023}\\
Event: Compute Seminar @ Technical University of Denmark (DTU).\\
 Host: Christian Gram Kalhauge.
\end{minipage}

\begin{minipage}{\textwidth}
\textbf{Executing Microservices on Serverless, Correctly}. \hfill {\em 2023}\\
Event: Sysread Seminar @ Brown University.\\
 Host: Shriram Krishnamurthi.
\end{minipage}

\begin{minipage}{\textwidth}
\textbf{Advancing the Serverless Paradigm}. \hfill {\em 2023}\\
Event: Invited Lecture at Systems Transforming Systems Course @ Brown University.\\
 Host: Nikos Vasilakis.
\end{minipage}

\begin{minipage}{\textwidth}
\textbf{PaSh: Practically Correct, Just-in-Time Shell Script Parallelization}. \hfill {\em 2023}\\
Event: Portland Programming Languages Seminar @ Portland State University.\\
 Host: Yao Li.
\end{minipage}

\begin{minipage}{\textwidth}
\textbf{Executing Microservices on Serverless, Correctly}. \hfill {\em 2023}\\
Event: Programming Languages Seminar @ Harvard University.\\
 Host: Stephen Chong.
\end{minipage}

\begin{minipage}{\textwidth}
\textbf{PaSh: Practically Correct, Just-in-Time Shell Script Parallelization}. \hfill {\em 2023}\\
Event: CSLab Computing Systems Day @ National Technical University of Athens.\\
 Host: Georgios Goumas.
\end{minipage}

\begin{minipage}{\textwidth}
\textbf{PaSh: Practically Correct, Just-in-Time Shell Script Parallelization}. \hfill {\em 2022}\\
Event: Invited Lecture at Systems Transforming Systems Course @ Brown University.\\
 Host: Nikos Vasilakis.
\end{minipage}

\begin{minipage}{\textwidth}
\textbf{PaSh: Practically Correct, Just-in-Time Shell Script Parallelization}. \hfill {\em 2022}\\
Event: New England Programming Languages and Systems Symposium (NEPLS) @ Harvard University.
\end{minipage}

\begin{minipage}{\textwidth}
\textbf{PaSh: Practically Correct, Just-in-Time Shell Script Parallelization}. \hfill {\em 2022}\\
Event: New Jersey Programming Languages and Systems Seminar (NJPLS) @ Stevens University.
\end{minipage}

\begin{minipage}{\textwidth}
\textbf{PaSh: Practically Correct, Just-in-Time Shell Script Parallelization}. \hfill {\em 2022}\\
Event: Languages, Systems, and Data Group Seminar @ University of California Santa Cruz.\\
 Host: Lindsey Kuper.
\end{minipage}

\begin{minipage}{\textwidth}
\textbf{PaSh: Data-parallel shell scripting}. \hfill {\em 2022}\\
Event: Programming Research Laboratory Seminar @ Northeastern University (Virtual).\\
 Host: Arjun Guha.
\end{minipage}

\begin{minipage}{\textwidth}
\textbf{Flumina: Correct Distribution of Stateful Streaming Computations}. \hfill {\em 2020}\\
Event: Programming Languages Tea @ University of California San Diego.\\
 Host: Nadia Polikarpova.
\end{minipage}

\begin{minipage}{\textwidth}
\textbf{Flumina: Correct Distribution of Stateful Streaming Computations}. \hfill {\em 2019}\\
Event: Athens Programming Languages Seminar @ National Technical University of Athens.\\
 Host: Kostis Sagonas and Nikos Papaspirou.
\end{minipage}

\begin{minipage}{\textwidth}
\textbf{HiPErJiT: A Profile-Driven Just-in-Time Compiler for Erlang}. \hfill {\em 2018}\\
Event: Athens Programming Languages Seminar @ National Technical University of Athens.\\
 Host: Kostis Sagonas and Nikos Papaspirou.
\end{minipage}




\section{References}

\textbf{Rajeev Alur}\\
Zisman Family Professor, Department of Computer and Information Science, University of Pennsylvania

\textbf{Nikos Vasilakis}\\
Assistant Professor, Department of Computer Science, Brown University

\textbf{Vincent Liu}\\
Assistant Professor, Department of Computer and Information Science, University of Pennsylvania

\textbf{Sebastian Burckhardt}\\
Senior Principal Researcher, Microsoft Research


% \section{Languages}

% \textbf{Greek} (Native), \textbf{English} (C2), \textbf{Italian} (C1), \textbf{German} (B2)


\end{resume}
\end{document}
